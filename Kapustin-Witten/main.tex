\documentclass{article}

%常规
\usepackage{geometry}
\usepackage{tikz}
%符号
\usepackage{amsmath}
\usepackage{amssymb}
\usepackage{physics}
%基础设置
\geometry{b5paper}
\linespread{1.5}
%个性化符号
%\newcommand{\dd}{\mathrm{d}}
\newcommand{\ii}{\mathrm{i}}
\newcommand{\ee}{\mathrm{e}}
% Equation numbering
\numberwithin{equation}{section}

\usepackage{newtxtext}
\usepackage{newtxmath}


\title{A Tale of Langlands Duality}
\author{Yilu Shao\\ \footnotesize{\it Institut de Mathématiques de Bourgogne, Université de Bourgogne Franche-Comté,}\\ \footnotesize{\it 9 avenue Alain Savary, Dijon, France}}
\date{}

\begin{document}

\maketitle

\begin{abstract}
    A self-study record of the Kapustin-Witten paper \cite{Kapustin:2006pk} on gauge theory and geometric Langlands program.
\end{abstract}

\section{Electro-magnetic Duality in Electromagnetics}
We start with the very easy abelian case as a warm-up. 
\paragraph{Convention}
I generally works with the Lorentzian signature $-+++$ but for most case we use Riemannian signature $++++$. Greek indices $\mu,\nu,...$ runs over $0,1,2,3$. Latin indices $i,j,...$ runs over space-like coordinate $1,2,3$, while $0$ always stands for time-like coordinate.

\subsection{Classical Theory}
The classical electrodynamics enjoys a good property. Consider the vacuum Maxwell equation
\begin{equation}
\begin{aligned}
\label{maxwell1}
\operatorname{div} \vec{{B}} &=0 &\operatorname{div} \vec{E} &=0  \\
\operatorname{rot} \vec{E} &=-\frac{\partial \vec{B}}{\partial t} & \operatorname{rot} \vec{B} &=\frac{\partial \vec{E}}{\partial t}.
\end{aligned}
\end{equation}
It is invariant under the transform
\begin{equation}
\label{em1}
  (\vec{E},\vec{B})\mapsto(\vec{B},-\vec{E}).
\end{equation}
This is the primitive form of electro-magnetic duality.

In the fancy bundle language, it is a $\mathrm{U}(1)$ gauge field described by connection $A$ on a principal $\mathrm{U}(1)$ bundle $E$ over a four-manifold $X$. The equation of motion is
\begin{equation}
\label{maxwell2}
\dd \star F=0
\end{equation}
with $F=\dd A$ is the curvature 2-form and $\star$ is the Hodge dual. This corresponds to the left hand set of equations in \eqref{maxwell1}. The remaining two equations indicate that $\dd F=0$, by Bianchi identity.

When $X=\vvmathbb{R}^{3,1}$, \eqref{maxwell2} is invariant under a transformation
\begin{equation}
\label{em2}
F \mapsto F^{\prime}=\star F .
\end{equation}
This just resembles the electro-magnetic duality, in the sense that taking
\begin{equation}
E_i=F_{0 i}, \quad B_i=\frac{1}{2} \epsilon_{i j k} F_{j k},
\end{equation}
we can easily recover \eqref{em1}.

Let me try to explain the confusing sentences in Kapustin's lecture \cite{Kapustin:2009ygz}.
\begin{quote}
    $F$ determines the holonomy of $A$ around all contractible loops in $X$. If $\pi_1(X)$ is trivial, $F$ completely determines $A$, up to gauge equivalence. In addition, if $H^2(X)\neq 0$, $F$ satisfies a quantization condition: its periods are integral multiples of $2\pi$. The cohomology class of $F$ is the Euler class of $E$ (or alternatively the first Chern class of the associated line bundle).
\end{quote}
The first thing is that

\subsection{Quantization}
It is the case that this duality no longer exists on arbitary manifold $X$ other then $\vvmathbb{R}^{3,1}$, but when it comes to the quantum theory, some miracle happens, as I will demonstrate in the following.

Quantization of gauge field is given by the path integral
\begin{equation}
Z=\int \mathcal{D} A\, \ee^{\ii S(A)}
\end{equation}
with integration over all possible topologies of the bundle $E$. Here the action is the celebrated
\begin{equation}
S(A)=\frac{1}{2 e^2} \int_X F \wedge \star F+\frac{\theta}{8 \pi^2} \int_X F \wedge F.
\end{equation}
Note that the $\theta$-angle is a topological invariant and relies only on the topology. It doesn't affect the classical equation of motion. So if we write explicitly the sum over all isomorphism classes of $E$
\begin{equation}
Z=\sum_E \int \mathcal{D} A\, \ee^{\ii S(A)}.
\end{equation}
the $\theta$-angle tells us how to weigh contributions of different $E$.

In order to compute it, we perform a Wick-rotation, then
\begin{equation}
Z=\sum_E \int \mathcal{D} A\, e^{-S_{\mathsf{E}}(A)}
\end{equation}
with an Euclidean action 
\begin{equation}
S_{\mathsf{E}}(A)=\int_X\left(\frac{1}{2 e^2} F \wedge \star F-\frac{i \theta}{8 \pi^2} F \wedge F\right).
\end{equation}

Assuming that $X$ is simply-connected. We want to replace the integration over $A$ with integration over the space of closed 2-forms $F$, which will make our life easier later. Here we introduce a ``Lagrangian multipler'' (I guess Kapustin actually wants to say this)
\begin{equation}
Z=\int \mathcal{D} F \mathcal{D} B \exp \left(-S_{\mathsf{E}}+\ii \int_X B \wedge \dd F\right)
\end{equation}
the new field $B$ is a 1-form on $X$ (or \textit{dual connection}), so that integration over it produces the desired constraint $\delta(\dd F)=\prod_{x\in X}\delta(\dd F(x))$. This allows us to integrate over all (not necessarily closed) 2-forms $F$. 

Using our favourite Gaussian integral, we get
\begin{equation}
Z=\int \mathcal{D} B \exp \left(-\frac{1}{2 \hat{e}^2} \int_X G \wedge \star G+\frac{\ii \hat{\theta}}{8 \pi^2} \int_X G \wedge G\right)
\end{equation}
where $G=\dd B$ is the \textit{dual curvature}. The new coupling constants $\hat{e}^2$ and $\hat{\theta}$ are defined by
\begin{equation}
\frac{\hat{\theta}}{2 \pi}+\frac{2 \pi \ii}{\hat{e}^2}=-\left(\frac{\theta}{2 \pi}+\frac{2 \pi \ii}{e^2}\right)^{-1}
\end{equation}



\subsection{$S$-duality}

\section{Montonen-Olive Duality}

\nocite{Witten:2008ep,Frenkel:2005pa}
\bibliographystyle{amsalpha}
\bibliography{main.bib}

\end{document}
